%
% CLDD Proposal
%
% Author: Geoff Johnson
%

\documentclass[11pt]{article}

\usepackage[titles]{tocloft}
\usepackage{verbatim}
\usepackage[pdftex]{graphics,graphicx}
\usepackage[export]{adjustbox}
\usepackage{pdfpages}
\usepackage{float}
\usepackage{amssymb}
\usepackage{biblatex}
\usepackage{tikz}
\usetikzlibrary{arrows}
\bibliography{references}

\usepackage{hyperref}
\hypersetup{
  pdfauthor={Geoff Johnson},
  pdftitle={CLDD Project Proposal},
  pdfsubject={Project Proposal},
  pdfkeywords={CLDD Project Proposal},
  colorlinks=true,
  linkcolor=black,
  urlcolor=blue
}

% Packages used for appendices.
\usepackage{appendix}
\usepackage{listings}
\usepackage{color}
\definecolor{light-gray}{gray}{0.95}
\definecolor{listinggray}{gray}{0.9}
\definecolor{lbcolor}{rgb}{0.9,0.9,0.9}
\definecolor{light-blue}{rgb}{0.6,0.720,0.85}

% The default margins are too wide all the way around, reset them.
\setlength{\topmargin}{-.5in}
\setlength{\textheight}{9in}
\setlength{\oddsidemargin}{0in}
\setlength{\textwidth}{6.5in}

% Change paragraph formatting.
\setlength{\parindent}{0pt}
\setlength{\parskip}{2ex}

\begin{document}
\nocite{*}

  \title{
    Project Proposal \\
    Configurable Server for Control, Logging, and Data Acquisition
  }

  \author{
    Geoff Johnson
    %<student ID> \\ \\
    %<course>
  }

  \renewcommand{\today}{January 1, 2014}
  \maketitle
  \thispagestyle{empty}
  \newpage
  \mbox{}
  \thispagestyle{empty}

  \newpage
  \addtocounter{page}{-1}
  \pagenumbering{roman}
  \tableofcontents
  %\listoffigures
  %\listoftables
  %\lstlistoflistings

  \newpage
  \pagenumbering{arabic}

  % XXX fill in or omit sections as needed, for now just blasting ideas

  % The institution of submission requires that this document contain
  % information about the student and company that the project is by/for.
  \section{Background}
    \label{sec:bg}

    \subsection{Student}
      \label{sec:bg-student}

    \subsection{Company}
      \label{sec:bg-company}

    \subsection{Project}
      \label{sec:bg-project}

  % Describe the problem domain and issues at a high level.
  \section{Problem Description}
    \label{sec:desc}

    \subsection{Scope/Domain}
      \label{sec:desc-domain}

      \subsubsection{Configuration Specification}
        \label{sec:desc-domain-conf}

      \subsubsection{Modular Abstraction Allowing for Device Type Expansion}
        \label{sec:desc-domain-mod}

      \subsubsection{Instrument Measurement}
        \label{sec:desc-domain-instr}

      \subsubsection{Automated Process Control}
        \label{sec:desc-domain-ctrl}

  % The proposed treatment to the problem described.
  \section{Proposed Solution}
    \label{sec:soln}

    % Discuss the benefit the end user gains from the proposed system that they
    % are currently missing.
    \subsection{Business Value}
      \label{sec:soln-val}

    \subsection{Development Process Model}
      \label{sec:soln-model}

      \subsubsection{Analysis}
        \label{sec:soln-model-analysis}

      \subsubsection{Design}
        \label{sec:soln-model-design}

      \subsubsection{Implementation}
        \label{sec:soln-model-implementation}

      \subsubsection{Testing}
        \label{sec:soln-model-testing}

    % Existing technologies needed to realize a solution.
    \subsection{Technologies Required}
      \label{sec:soln-tech}

      \subsubsection{ZeroMQ and Protocol Buffers for Messaging}
        \label{sec:soln-tech-msg}

      \subsubsection{Configuration and Communication Through libCLD}
        \label{sec:soln-tech-cld}

    % XXX some other subsection ideas:
    %     - what metrics will be used to evaluate success
    %     - how will you know your objective has been met

  % Step-by-step details of the work required to achieve the proposed goal.
  \section{Work Plan}
    \label{sec:plan}

    \subsection{Project Timeline}
      \label{sec:plan-time}

    \subsection{Tasks \& Activities}
      \label{sec:plan-tasks}

    \subsection{Milestones}
      \label{sec:plan-milestones}

  % Insert a list of references that were cited.
  \newpage
  \printbibliography

  % Appendices
  \newpage
  \addappheadtotoc
  \appendix
  \appendixpage

  % Left over from a previous proposal document, edit later as needed.
  \section{Glossary of Abbreviations and Terms}
    \label{app:glossary}

    Commonly used terms and an explanation of them are given in Table
    \ref{tab:glossary}, these are primarily generic computing terms.

    \begin{table}[H]
      \centering
      \begin{tabular}{l | p{10cm}}
        \hline
        Term & Definition/Explanation \\ [0.5ex]
        \hline\hline
        Advantech & An industrial data acquisition device manufacturer. \\
        C & Common software programming language. \\
        Client & Refers to the software application that communications with a daemon. \\
        Comedi & Open source hardware drivers for Control and Measurement Devices. \\
        Daemon & Common term used to refer to a server application in a Linux system. \\
        Data Acquisition & The act of gathering data from a real world process. \\
        GLib & Standard set of Linux system libraries for the Gnome window manager. \\
        GNU & A collection of applications, libraries, and developer tools. \\
        Perl & A high level programming language good for rapid development. \\
        Python & A high level programming language good for rapid development. \\
        RS232 & Serial communication devices that has been ubiquitous for decades. \\
        Vala & An object oriented programming language that uses the GObject type system and compiles to C code. \\
        Watchdog & Standard concept for monitoring a vital systems heartbeat. \\
        XML & Extensible markup language, common for use in messaging systems. \\
        \hline
      \end{tabular}
      \caption{Glossary of Terms}
      \label{tab:glossary}
    \end{table}

\end{document}
